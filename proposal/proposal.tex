\documentclass{article}

\usepackage{hyperref}

\title{Criminal stuff}
\author{Kenneth Lin, Thomas McCormick, Siddharth Naik}
\date{\today}

\begin{document}
\maketitle

For our project, we would like to focus on the large breadth of public
safety data available at DataSF
(\url{https://data.sfgov.org/data?category=Public\%20Safety}). We believe
that the quality and sheer volume of data on crime, fire hazards, and other
incidents will allow us to create an interesting and insightful project.

While we have not yet narrowed down specific areas of interest, we have a
few tentative projects in mind. For example, one such project would be to
investigate the correlation between locations of crime incidents and the
level of wealth in a neighborhood. It is well known that low-income areas
are more susceptible to crime -- when income alone does not suffice to
provide for households, people (mostly youth) turn to crime. In addition to
basic visualizations with graphs of average income level vs. crime rate, or
maps of crime incidents alongside income level, we could also paint an
``average'' portrait of high-crime areas according to the data. To do this,
we would collect photos of, for example, a street corner in San Francisco,
and classify the street corner as high- or low-crime according to the
DataSF data. We then take the ``average'' of all such street corners in
each given area according to its level of crime and thus produce a
photograph of what the ``average, high-crime'' neighborhood looks
like. Interesting visualizations like this and more are possible with such
a varied and large compendium of data.

Another possible, even more ambitious project, is to thoroughly analyze the
the times and places of vehicular burglary and predict either the location
of crime organizations or the time and place of future such crimes. This
is, of course, under the assumption that organized criminal groups
exist. However, given some personal experience in addition to many
anecdotes, it is not unreasonable to assume these may exist. If it were
possible to detect patterns in this data on either a micro or macro scale,
then this would be a huge boon to SFPD and public safety overall in San
Francisco. The end result would either be a list of locations where
criminal groups are likely to exist, or a system that can be used by the
police department to aid in stopping crime before it occurs.

\end{document}
